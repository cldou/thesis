\begin{center}
{\textbf{Abstract}}
\end{center}
\addcontentsline{toc}{chapter}{Abstract}

Integrated sensing and communication (ISAC) has emerged as a promising para\-digm for future wireless networks, enabling spectrum-efficient dual-functional services for emerging applications that require both high-throughput data transmission and high-precision environmental sensing. However, realizing the full potential of ISAC is fundamentally constrained by resource bottlenecks across multi-dimensional domains: the rigidity of static time-slot allocation in the time domain, the saturation of orthogonal access in the frequency domain, the inequality of multi-node cooperation in the spatial domain, and the siloed operation among sensing, communications and computing in the functional domain. In this thesis, we explore four resource-efficient designs to systematically address these challenges across time, frequency, spatial, and functional domains. Specifically, we focus on the joint optimization of beamforming, scheduling, and multi-dimensional resource allocation to improve the overall system-wise resource efficiency. Firstly, we propose a channel-sharing aided ISAC scheme that overcomes the rigidity of static time-slot allocation by enabling on-demand dynamic scheduling for both sensing and communication. Secondly, we introduce a non-orthogonal multiple access (NOMA)-aided ISAC framework with sensing scheduling to unlock the spectral capacity limit. Thirdly, we present a fairness-aware cooperative sensing strategy to eliminate spatial blind spots and ensure network longevity. Lastly, we design a two-tier integrated sensing, communications, and computing (ISCC) framework to achieve the closed-loop resource optimization for energy-efficient multi-service coexistence. This thesis investigates diverse resource-efficient optimization frameworks to enhance the performance of ISAC systems, laying the foundation for a paradigm shift from simple functional coexistence to deep intelligent integration, thereby ensuring robust, scalable, and green network operations in diverse scenarios. The main works and innovations are summarized as follows.

\begin{itemize}	
    \item In our first work, we propose an energy-efficient ISAC scheme that leverages dynamic channel sharing to break the rigidity of static time-slot allocation. The proposed approach allows the base station dynamically reuses the downlink communication channels of cellular users to perform radar sensing towards multiple targets. We formulate an optimization problem aimed at maximizing the energy efficiency of the radar sensing while strictly guaranteeing the communication users' quality of service (QoS). We jointly optimize the sensing scheduling and transceiver beamforming to alleviate the co-channel interference introduced by the channel reuse. An efficient alternating optimization framework combined with Dinkelbach's method is proposed to address the non-convex fractional programming problem. 

    \item In the second work, we address the spectral bottleneck and capacity limitations of orthogonal multiple access (OMA) in multi-user scenarios by proposing a NOMA-aided ISAC framework. The base station transmits superimposed signals to serve multiple NOMA users while simultaneously leveraging these signals as probing waveforms for multi-target sensing. We formulate an optimization problem to maximize the sensing efficiency, which is a novel metric defined as the number of successfully sensed targets per time unit, by jointly optimizing the beamforming, the NOMA transmission duration, and the sensing scheduling. We design a decomposition-based algorithm by utilizing successive convex approximation (SCA) and the penalty function method to solve the mixed-integer non-convex problem, significantly enhancing the system's connection density and concurrent sensing capability.

    \item In our third work, we propose a fairness-aware ISAC-enabled cooperative sensing strategy to overcome single-node blind spots and the inequality of cooperation, where distributed devices access the channel in a time-division manner, with each device performing environmental sensing and data transmission simultaneously during its assigned time slot. To address the ``wooden barrel effect" in cooperative networks and prolong the network lifetime, we formulate an optimization problem aimed at maximizing the fairness-aware system-wise throughput. To tackle the non-convexity, we design a block coordinate descent (BCD)-based algorithm to decompose the problem into beamforming and time allocation subproblems. Specifically, we analyze the necessity of dedicated sensing signals and employ SCA to optimize the beamforming vectors. Furthermore, we characterize the features of the optimal time allocation to derive a semi-analytical expression, ensuring an efficient and balanced resource allocation.

    \item In the final work, we design an integrated sensing and two-tier task offloading framework to address the conflict of multi-service coexistence and bridge the functional silo between sensing and computing. Specifically, we investigate a two-tier setup that facilitates NOMA-assisted task offloading for edge users while the base station performs continuous radar sensing. To counteract the resource competition between high-power sensing and latency-sensitive computing, we propose a cross-functional synergy strategy that employs tier-specific signaling to decouple mutual interference. We establish an optimization problem aimed at minimizing the total energy consumption by jointly optimizing the transmit beamforming, the two-tier dedicated sensing signals, the two-tier computation offloading strategies and the associated allocations of the communication and computing resources. Although the formulated joint optimization problem is non-convex, we identify the features of its solutions and exploit a decomposition-based framework for solving it, realizing a holistic functional closed-loop design that supports computation-intensive applications alongside ubiquitous sensing.
\end{itemize}

