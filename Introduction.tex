%% \documentclass[12pt,twoside,a4paper]{book}
\usepackage[numbers,square,sort&compress]{natbib}
\usepackage{epsfig,epstopdf}

\usepackage{multirow}
\usepackage{latexsym}
\usepackage{makecell}
\usepackage{graphicx,amsmath,amsfonts,amssymb,amscd,amsthm}
%\usepackage{stfloats,float,subfig, subfloat}
\usepackage{amsmath}
\usepackage{amsfonts}
\usepackage{amssymb}
\usepackage{bm}
\usepackage{threeparttable,booktabs}
\usepackage{epic}
\usepackage{rotating}
\usepackage{amsthm}
\usepackage[compact]{titlesec}
\usepackage{times}
\usepackage{titletoc}
\usepackage{titlesec}
\usepackage{indentfirst}
\usepackage{color}
\usepackage{caption,tikz-qtree}
\usepackage{amsmath,amssymb,epsfig,color,graphicx,url,amsthm,mathtools}
\usepackage{float}
\usepackage{clrscode3e}
\usepackage{nomencl}

\newcommand{\lotlabel}{Table}
\newcommand{\loflabel}{Figure}
\usepackage{amsmath}
\usepackage{multirow}
\usepackage{diagbox}
\usepackage{longtable}
\usepackage{graphicx}
\usepackage{CJK}
\usepackage{mathrsfs}
\usepackage{times}
\usepackage{amsmath,amssymb,latexsym}
\usepackage{epsfig}
\usepackage{tabularx}
\usepackage{ifpdf}
\usepackage{algorithmic}
\usepackage{algorithm}
\usepackage{array}
\usepackage{mdwmath}
\usepackage{CJK}
\usepackage{mdwtab}
\usepackage{eqparbox}
\usepackage{subfigure}
\usepackage{color}
\usepackage{url}
\usepackage{graphicx} %use graph format
\usepackage{epstopdf}
\usepackage{morefloats}
\usepackage{subfigure}
\usepackage[font=small]{caption}
\usepackage{mathrsfs}
\usepackage{times}
\usepackage{amsmath,amssymb,latexsym}
\usepackage{epsfig}
\usepackage{tabularx}
\usepackage{ifpdf}
\usepackage{algorithmic}
\usepackage{algorithm}
\usepackage{array}
\usepackage{bm}
\usepackage{mdwmath}
\usepackage{mdwtab}
\usepackage{eqparbox}
\usepackage{subfigure}
\usepackage[justification=centering]{caption}
%\usepackage{float}
%\usepackage{subfloat}
\usepackage{color}
\usepackage{slashbox}
\usepackage{url}
\usepackage{textcomp}
\usepackage{color}
\usepackage{subfigure}
\usepackage{mathrsfs}
\usepackage{times}
\usepackage{amsmath,amssymb,latexsym}
\usepackage{epsfig}
\usepackage{tabularx}
\usepackage{ifpdf}
\usepackage{algorithmic}
\usepackage{algorithm}
\usepackage{array}
\usepackage{mdwmath}
\usepackage{mdwtab}
\usepackage{eqparbox}
\usepackage{subfigure}
\usepackage{color}
\usepackage{url}
\usepackage{amsthm}
\usepackage{morefloats}
\usepackage{float}
\usepackage{booktabs}
\makeatletter
\def\ps@headings{%
\def\@oddhead{\mbox{}\scriptsize\rightmark \hfil \thepage}%
\def\@evenhead{\scriptsize\thepage \hfil \leftmark\mbox{}}%
\def\@oddfoot{}%
\def\@evenfoot{}}
\newcommand{\biggg}{\bBigg@{3}}
\def\bigggl{\mathopen\biggg}
\def\bigggm{\mathrel\biggg}
\def\bigggr{\mathclose\biggg}
\newcommand{\Biggg}{\bBigg@{3.5}}
\def\Bigggl{\mathopen\Biggg}
\def\Bigggm{\mathrel\Biggg}
\def\Bigggr{\mathclose\Biggg}
\newcommand{\bigggg}{\bBigg@{4}}
\def\biggggl{\mathopen\bigggg}
\def\biggggm{\mathrel\bigggg}
\def\biggggr{\mathclose\bigggg}
\newcommand{\Bigggg}{\bBigg@{4.5}}
\def\Biggggl{\mathopen\Bigggg}
\def\Biggggm{\mathrel\Bigggg}
\def\Biggggr{\mathclose\Bigggg}
\newcommand{\biggggg}{\bBigg@{5}}
\def\bigggggl{\mathopen\biggggg}
\def\bigggggm{\mathrel\biggggg}
\def\bigggggr{\mathclose\biggggg}
\newcommand{\Biggggg}{\bBigg@{5.5}}
\def\Bigggggl{\mathopen\Biggggg}
\def\Bigggggm{\mathrel\Biggggg}
\def\Bigggggr{\mathclose\Biggggg}
\makeatother

\makenomenclature
%\usepackage[ruled,linesnumbered]{algorithm2e}
%\usepackage{algorithm2e}
%\usetikzlibrary{matrix,arrows}


%\usepackage{multirow}
%\usepackage{diagbox}
%\usepackage{longtable}
%\usepackage{CJK}
%\usepackage{mathrsfs}
%\usepackage{mdwmath}
\usepackage{graphicx}
%\usepackage{epstopdf}
%\usepackage[font=small]{caption}
%\usepackage[justification=centering]{caption}
%\usepackage{textcomp}
%\usepackage{color}
\usepackage{subfigure}
%\usepackage{cite}
%\usepackage{times}
%\usepackage{amsmath,amssymb,latexsym}
%\usepackage{epsfig}
%\usepackage{tabularx}
%\usepackage{ifpdf}
\usepackage{algorithmic}
\usepackage{algorithm}
%\usepackage{array}
%\usepackage{mdwtab}
%\usepackage{eqparbox}
%\usepackage{color}
%\usepackage{url}
%\usepackage{amsthm}
%\usepackage{morefloats}
%\usepackage{float}
\usepackage{cases}
%\usepackage{threeparttable,booktabs}

\newtheorem{assumption}{Assumption}
\newtheorem{property}{Property}




\usepackage{fancyhdr}%set page foot and page head

\usepackage{graphicx}
\graphicspath{{./}{figs_iotj/}{figs_tetc/}{figs_tvt/}{figs_tgcn/}{figs_other/}}
\usepackage{epstopdf,enumerate,float,leftidx,booktabs}
\usepackage[font=small,labelfont=bf]{caption}

%%%%%%%%%%%%%%%%% in phdthesis.sty %%%%%%%%%%%%%%%%%%%%%%%%
\setlength{\footskip}{0.8cm} \setlength{\headheight}{0.5cm}
\newlength{\textheit}
\setlength{\textheit}{\paperheight} \addtolength{\textheit}{-2.0in}
\addtolength{\textheit}{-\topmargin}
\addtolength{\textheit}{-\headheight}
\addtolength{\textheit}{-\headsep}
\addtolength{\textheit}{-\footskip}
\setlength{\textheight}{\textheit} \flushbottom
\renewcommand{\contentsname}{\centerline Table of Contents}
\renewcommand{\chaptername}{Chapter~}
\usepackage[left=4cm,right=2.55cm,top=2.35cm,bottom=2.65cm]{geometry}
\geometry{a4paper,scale=0.99}
\titlecontents{chapter}[0pt]{\vspace{0.1\baselineskip}\bfseries}
    {\bfseries Chapter~\thecontentslabel~\quad}{}
    {\hspace{0.5em}\titlerule*[10pt]{$\cdot$}\contentspage}



\titlecontents{section}[0pt]{\vspace{0.1\baselineskip}}
    {\thecontentslabel~\quad}{}
    {\hspace{0.5em}\titlerule*[10pt]{$\cdot$}\contentspage}

   \titlecontents{subsection}[0pt]{\vspace{0.1\baselineskip}}
    {~~~\qquad\thecontentslabel~\quad}{}
    {\hspace{0.5em}\titlerule*[10pt]{$\cdot$}\contentspage}


\titleformat{\chapter}[display]
{\normalfont\bfseries}{\chaptertitlename\ \thechapter}{12pt}{}
%{\Large\bfseries}{\chaptertitlename\ \thechapter}{12pt}{}


\titleformat{\section}[hang]{\bfseries}{\thesection}{1em}{}
\titleformat{\subsection}[hang]{\bfseries}{\thesubsection}{1em}{}



\titlespacing*{\chapter}{0pt}{-1.1cm}{*1}

\let\cleardoublepage\clearpage


\usepackage{natbib}
\newcommand{\gbold}[1] {\mbox{\boldmath${#1}$\unboldmath}}
\newcommand{\Smatrix}[1] {\left[ \begin{array}{ccc} #1\end{array} \right] }
\newcommand{\smatrix}[1] {\left[ \matrix{#1} \right] }
\newtheorem{theorem}{Theorem}[section]\large
\newtheorem{lemma}{Lemma}[section]
\newtheorem{axiom}{Axiom}[section]
\newtheorem{definition}{Definition}[section]
\newtheorem{observation}{Observation}[section]
\newtheorem{corollary}{Corollary}[section]
\newtheorem{proposition}{Proposition}
\newtheorem{example}{Example}[section]
\newtheorem{remark}{Remark}[section]
%\newtheorem{algorithm}{Algorithm}[section]
\newtheorem{find}{Finding}
%\newcommand{\eproof}{\hfill\rule{2.2mm}{3.0mm}}
%\renewcommand{\proofname}{\bf Proof}
%\renewcommand{\algorithmicrequire}{\textbf{Input:}} % Use Input in the format of Algorithm
%\renewcommand{\algorithmicensure}{\textbf{Output:}} % Use Output in the format of Algorithm
\newtheorem{thm}{Theorem}
\newtheorem{lem}[thm]{Lemma}
\newtheorem{dfn}{Definition}
\newtheorem{THM}{$\mathbf{Theorem}$}
\newtheorem{LKSVD}{$\mathbf{Lemma}$}
\newcommand{\tbfx}{\textbf{x} }
\newtheorem{Lemma}{$\mathbf{Lemma}$}
\newtheorem{Assumption}{$\mathbf{Assumption}$}
\newtheorem{pro1}{$\mathbf{Proposition}$}
\newtheorem{coro}{$\mathbf{Corollary}$}
\newcommand{\FB}{\mathfrak{B}}
%\newtheorem{theorem}{$\mathbf{Theorem}$}
\newtheorem{assume}{$\mathbf{Assumption}$}
%\newtheorem{lemma}{$\mathbf{Lemma}$}
\newcommand{\eg}{\textit{e}.\textit{g}.}

\def\x{{\mathbf x}}
\def\L{{\cal L}}
\newcommand{\BEta}{\boldsymbol{\eta}}
\newcommand{\CF}{\mathcal{F}}
\newcommand{\CX}{\mathcal{X}}
\newcommand{\CS}{\mathcal{S}}
\newcommand{\Bx}{\bm{x}}
\newcommand{\Beta}{\bm{\eta}}
\renewcommand{\figurename}{Fig.}
\renewcommand{\proofname}{\bf Proof}
\newcommand{\redcolor}{\textcolor{red}}
\newcommand{\CQ}{\mathcal{Q}}
\newcommand{\TZ}{\tilde{Z}}
\newcommand{\Hy}{\hat {y}}
\newcommand{\HP}{\hat {\mathbf{P}}}
\newcommand{\HC}{\hat {\mathbf{C}}}
\newcommand{\Tz}{\tilde{z}}
\newcommand{\CR}{\mathcal{R}}
\newcommand{\CG}{\mathcal{G}}
\newcommand{\BZ}{\mathbf{Z}}
\newcommand{\BR}{\mathbf{R}}
\newcommand{\hc}{\hat{c}}
\newcommand{\Hm}{\hat{m}}
\newcommand{\BF}{\mathbf{F}}
\newcommand{\ta}{\theta}
\newcommand{\Bs}{\mathbf{s}}
\newcommand{\Bu}{\mathbf{u}}
\newcommand{\Bv}{\mathbf{v}}
\newcommand{\Bc}{\mathbf{c}}
\newcommand{\BB}{\mathbf{B}}
\newcommand{\BA}{\mathbf{A}}
\newcommand{\Bh}{\mathbf{h}}
\newcommand{\Bf}{\bm{f}}
\newcommand{\by}{\mathbf{y}}
\newcommand{\BX}{\mathbf{X}}
\newcommand{\BD}{\mathbf{D}}
\newcommand{\Bp}{\mathbf{p}}
\newcommand{\BY}{\mathbf{Y}}
\newcommand{\BS}{\mathbf{S}}
\newcommand{\BAT}{\breve{\mathbf{\Theta}}}
\newcommand{\BW}{\mathbf{W}}
\newcommand{\BK}{\mathbf{K}}
\newcommand{\BI}{\mathbf{I}}
\newcommand{\Be}{\mathbf{e}}
\newcommand{\Bw}{\mathbf{w}}
\newcommand{\Bsg}{\bm{\mu}}
\newcommand{\Bmu}{\bm{\sigma}}
\newcommand{\BL}{\mathbf{L}}
\newcommand{\BU}{\mathbf{U}}
\newcommand{\BV}{\mathbf{V}}
\newcommand{\BC}{\mathbf{C}}
\newcommand{\Bb}{\mathbf{b}}
\newcommand{\BE}{\hat{\mathbf{E}}}
\newcommand{\HW}{\hat{\mathbf{W}}}
\newcommand{\BT}{\mathbf{\Theta}}
\newcommand{\Br}{\mathbf{r}}
\newcommand{\BP}{\mathbf{P}}
\newcommand{\CC}{\mathcal{C}}
\newcommand{\CK}{\mathcal{K}}
\newcommand{\CU}{\mathcal{U}}
\newcommand{\CP}{\mathcal{P}}
\newcommand{\Tt}{\tilde{\theta}}
\newcommand{\tz}{\tilde{z}}
\newcommand{\Hw}{\hat{w}}
%\newcommand{\CD}{\mathcal{D}}
\newcommand{\CA}{\mathcal{A}}
\newcommand{\CL}{\mathcal{L}}
\newcommand{\TT}{\tilde{\mathbf{\Theta}}}
\newcommand{\HT}{\hat{\mathbf{\Theta}}}
\newcommand{\tabincell}[2]{
	\begin{tabular}{@{}#1@{}}#2\end{tabular}
}


\usepackage{setspace}%set double line-space
\renewcommand{\baselinestretch}{1.6}  %set double line-space

\allowdisplaybreaks[3] %%allow the equatins change to the next page

\pagestyle{plain}%foot is the page number

%\begin{document}

\chapter{Introduction}\label{chapter_introduction}

\section{Research Background}
\label{chap1_sec_background}
\setlength{\parindent}{2ex}

The evolution of wireless networks towards the sixth-generation (6G) era marks a fundamental paradigm shift from ``connecting things" to ``perceiving the world" \cite{intro.9390169,intro.8808168,intro.9509294}. Future networks are envisioned to serve as intelligent ecosystems that integrate the physical and digital realms. To support emerging immersive applications such as autonomous driving and digital twins, the network are required to simultaneously provide ultra-reliable communication and high-precision environmental sensing \cite{intro.9540344,intro.9330512,intro.9585321}. In response to this dual demand, integrated sensing and communication (ISAC) has emerged as a key enabling technology. By unifying radar sensing and wireless communication into a single hardware platform and sharing the same radio spectrum, ISAC breaks the traditional separation of two independent systems \cite{intro.10418473,intro.9924202,intro.9705498}. This integration promises significant gains in spectral efficiency and reduced hardware costs, establishing itself as a foundational pillar of next-generation wireless infrastructure.

However, the transition from functional coexistence to deep integration introduces severe challenges regarding resource efficiency. Integrating two functions into one system implies that sensing and communication compete for the same pool of already scarce radio resources \cite{intro.9606831,intro.9393464,intro.10529727}. Radar sensing typically prioritizes high-power, short-duration pulses for accurate detection, whereas communication systems prioritize continuous transmission for high throughput. When these conflicting objectives share the same resource pool, naive resource sharing often leads to mutual interference and performance degradation \cite{intro.8999605}. Consequently, the central challenge in designing sustainable ISAC systems is how to optimally orchestrate limited resources to achieve a balance between sensing and communication.

This challenge is further exacerbated as network environments become increasingly versatile and heterogeneous. Current resource management frameworks, primarily designed for single-functional systems, struggle to cope with the multi-dimensional complexity of future services. The structural limitations of existing architectures manifest as distinct ``resource bottlenecks" across four key domains:
\begin{itemize}
    \item \textit{Time Domain:} Traditional static time-slot allocation leads to rigid partitioning between sensing and communication functions. It fails to adapt to the varying temporal requirements, such as the bursty nature of data traffic versus the periodic nature of sensing tasks, resulting in inefficient resource utilization \cite{intro.9420261,intro.9728752}.
    \item \textit{Frequency Domain:} With the exponential growth of wireless devices, the electromagnetic spectrum is becoming saturated. Conventional orthogonal access schemes face a spectral capacity limit, restricting the system's ability to support dense connectivity and high-resolution sensing simultaneously \cite{intro.10736660,intro.10255745}.
    \item \textit{Spatial Domain:} Single-node sensing is inherently limited by physical blockage and coverage blind spots. Although cooperative strategies can mitigate this, they introduce systemic issues regarding spatial unfairness and uneven resource depletion among distributed nodes \cite{intro.10540249,intro.10726912}.
    \item \textit{Functional Domain:} As networks evolve to support diverse concurrent services beyond simple data transfer or environmental sensing, the functional isolation between different subsystems leads to resource silos. The lack of a unified coordination mechanism results in severe competition and inefficiency when multiple functions coexist \cite{intro.10375321,intro.11185116}.
\end{itemize}

To address these challenges, this thesis is dedicated to investigating systematic resource-efficient optimization frameworks for ISAC systems. Firstly, we propose a channel-sharing aided ISAC scheme to break the rigidity of static time-slot allocation in the time domain. Secondly, we introduce a non-orthogonal multiple access (NOMA)-aided ISAC framework to unlock the spectral capacity limit in the frequency domain. Thirdly, we present a fairness-aware cooperative sensing strategy to eliminate spatial blind spots and ensure network longevity in the spatial domain. Lastly, we design a two-tier integrated sensing, communications, and computing (ISCC) framework to achieve the closed-loop resource optimization for energy-efficient multi-service coexistence. By jointly optimizing beamforming, scheduling, and multi-dimensional resource allocation, we aim to ensure robust, scalable, and sustainable network operations in diverse scenarios.


\section{Literature Review}
\label{chap1_sec_Review}

This section reviews the existing literature on ISAC, focusing on resource efficiency across the time, frequency, spatial, and functional domains. To systematically identify the technical limitations in current research, we categorize the related works into four subsections: resource allocation and scheduling, multiple access techniques, networked cooperative sensing, and ISCC. Each subsection analyzes the state-of-the-art strategies and highlights the specific research gaps that motivate the contributions of this thesis.

\subsection{Resource Allocation and Scheduling in ISAC}
\label{chap1_ssec_1}

The fundamental challenge in ISAC systems lies in the optimal management of limited radio resources to support the dual functionalities of sensing and communication. Existing research in this domain can be generally categorized into joint beamforming design and systemlevel resource scheduling. Regarding the joint beamforming design, a substantial body of literature focuses on characterizing the trade-off between sensing and communication performance. In \cite{twc1.9868348}, Wang \textit{et. al.} investigated the partially-connected hybrid beamforming design for multi-user ISAC systems, aiming at minimizing the Cram\'er-Rao bound while satisfying the signal-to-interference-plus noise ratio constraint for individual communication user. In \cite{twc1.9842350}, Huang \textit{et. al.} studied the coordinated power control among ISAC transmitters for communication and distributed radar sensing. To further optimize resource utilization, Dong \textit{et. al.} investigated the power and bandwidth resources allocation in ISAC systems by considering the fairness of the sensing service in \cite{twc1.9945983}. Similarly, in \cite{tvt.9761984}, He \textit{et al.} investigated the energy-efficiency optimization of the multi-user communication in ISAC system while guaranteeing the sensing requirement. Focusing on the sensing resolution, a bandwidth allocation strategy has been proposed in \cite{tvt.9796610} to optimize the weighted average range resolution for sensing while guaranteeing the sum-rate among user equipments. In \cite{twc2.liu2022optimal}, Liu \textit{et. al.} proposed a single-target-multi-beams radar beam alignment scheme to obtain more accurate estimation information. In \cite{twc2.he2023full}, He \textit{et. al.} investigated the joint optimization of a full-duplex communication based ISAC system to improve spectral efficiency.

Beyond the physical layer design, there have been several studies leveraging ISAC in diverse network paradigms. In \cite{twc1.10107972}, Huang \textit{et. al.} investigated the trade-off between communication performance and sensing performance in unmanned aerial vehicle (UAV)-assisted ISAC systems. Extending this to ground network enhancement, Yang \textit{et. al.} exploited ISAC UAVs for improving both communication and localization performances of the ground networks in \cite{twc1.10122889}. Addressing security concerns, a secure precoding optimization scheme has been proposed in \cite{tvt.9927490} for NOMA-ISAC networks. Moreover, in \cite{twc2.cui2023physical}, Cui \textit{et. al.} established a physical layer framework for digital twin based on the ISAC and proposed a dual-functional waveform augmentation strategy.

More recently, the focus has shifted towards multi-target sensing and dynamic scheduling. In \cite{twc1.9390402}, the spectrum allocation problem between spectrum service providers and terminals equipped with orthogonal frequency division multiplexing ISAC system is studied. Considering the temporal dynamics, Yang \textit{et. al.} investigated the queue-aware dynamic resource scheduling for the ISAC system while taking the network stability and radar detection performance into account in \cite{twc1.9282063}. For specific applications like vehicular tracking, Wang \textit{et. al.} developed an approach for multi-vehicle tracking and identity association by using the ISAC signals in \cite{twc1.9820762}, which reduces the communication overhead and latency. Additionally, in \cite{twc1.9124713}, Liu \textit{et. al.} proposed a joint transmitting beamforming model for ISAC systems, with the objective of maximizing the radar transmitting beamforming performance while guaranteeing the communication quality. 

However, most existing works predominantly adopt a static orthogonal partitioning strategy, where sensing and communication are isolated in fixed time slots. Such rigid frameworks fail to adapt to the time-varying and asymmetric nature of service demands, where communication traffic is typically bursty while sensing tasks are periodic, leading to low resource utilization. Consequently, there is a lack of research on dynamic scheduling paradigms that can flexibly exploit the temporal degrees of freedom to coordinate these conflicting tasks. This indicates a critical requirement for a demand-responsive resource management framework to break the temporal rigidity, which motivates the design proposed in Chapter 2.

\subsection{Multiple Access Techniques for ISAC}
\label{chap1_ssec_2}

ISAC has attracted wide research interests in integrated signal design and resource management owing to its potential advantages. In the context of orthogonal multiple access (OMA), Wu \textit{et. al.} exploited the multiple input multiple output (MIMO)-orthogonal frequency division multiplexing (OFDM) data symbols carried by subcarriers for time-domain and spatial-domain signal orthogonality for ISAC in \cite{iotj.10298608}. To enhance sensing accuracy, Liu \textit{et. al.} proposed a radar beam alignment scheme in \cite{iotj.9849103} for acquiring accurate estimation information in vehicle-to-everything systems by allocating multiple radar beams to the target. Furthermore, in \cite{iotj.10294279}, Wang \textit{et. al.} proposed an ISAC with computation over-the-air framework to improve the spectrum efficiency and quality of service in wireless sensor networks. Leveraging artificial intelligence, Liu \textit{et. al.} investigated the predictive ISAC beamforming in vehicular networks in \cite{iotj.liu2022learning} where the deep learning is employed to learn historical channels' features and predict the beamforming matrix for the next time slot.

To further improve spectral efficiency and connectivity, NOMA has been regarded as a key enabling technology for future B5G/6G networks. Prior to its integration with ISAC, NOMA has been extensively studied. In \cite{tvt.8988182}, Zhao \textit{et. al.} investigated the security of unmanned aerial vehicle and NOMA aided networks via power allocation and beamforming. In \cite{tvt.9915477}, Li \textit{et. al.} studied the secrecy performance of the simultaneously transmitting and reflecting reconfigurable intelligent surface assisted NOMA networks. In \cite{tvt.10050446}, Li \textit{et. al.} proposed an adaptive multi-user association strategy for NOMA-aided visible light communication systems. 

Recently, integrating NOMA with ISAC has emerged as a promising direction. In \cite{tvt.wang2022noma}, Wang \textit{et al.} studied the NOMA-empowered ISAC system to maximize the weighted sum of communication throughput and effective sensing power. From the perspective of bandwidth partition, Zhang \textit{et. al.} investigated the NOMA assisted ISAC networks in \cite{tvt.10036107}, where a portion of the bandwidth is used for ISAC and the remaining bandwidth is used for wireless communication only. Furthermore, in \cite{tvt.9996408}, the resource allocation in NOMA-aided joint communication, sensing, and multi-tier computing has been studied. 

However, the majority of existing NOMA-ISAC frameworks implicitly assume a ``sense-all" strategy, where resources are distributed to sense all potential targets simultaneously regardless of their priorities or channel conditions. In dense scenarios, such an indiscriminate approach often leads to resource fragmentation and degraded detection performance for critical targets. Consequently, there is a lack of research on joint optimization frameworks that integrate user access control with target selection mechanisms. This highlights the necessity for a spectral-efficient design that can balance the trade-off between massive connectivity and high-quality multi-target sensing through intelligent scheduling, which motivates the sensing-efficient NOMA-ISAC framework proposed in Chapter 3.

\subsection{Networked and Cooperative ISAC}
\label{chap1_ssec_3}

Single-node ISAC systems often face limitations in sensing coverage and resolution. Consequently, there have been several efforts on multi-target sensing and waveform design to mitigate these issues.
In \cite{iotj.10121683}, Ma \textit{et. al.} proposed a max-aperture radar slicing waveform to yield a large time-frequency aperture for multi-user communication and multi-target sensing. In \cite{iotj.hua2023optimal}, Hua \textit{et. al.} designed the radar sensing signal in the downlink ISAC system where the base station performs multi-user communication and radar sensing simultaneously. In \cite{iotj.du2023multi}, the transceiver waveform is designed for a general scenario of multi-user multi-target ISAC systems that the communication users can be simultaneously served and detected. In \cite{iotj.10109100}, the authors proposed an ISAC framework established on the Markov decision process and deep reinforcement learning to enable the adaptive beamforming of autonomous vehicles for multi-target sensing. 


To fundamentally address the insufficient sensing accuracy of an individual device caused by resource constraints and blockage, there have been several studies in exploring the potential of multi-device cooperative sensing. In \cite{iotj.9282206}, Chen \textit{et. al.} investigated a beam sharing assisted cooperative sensing UAV networks and presented upper-bound average cooperative sensing area as the metric for cooperative sensing UAV networks. In the context of vehicular networks, Cheng \textit{et. al.} investigated the multi-vehicle multi-sensor cooperative tracking in vehicular communication networks in \cite{iotj.9830717} with the multi-model sensing information sharing and fusion. Expanding to inference tasks, in \cite{iotj.10478867}, Li \textit{et. al.} investigated the device-edge co-inference in a wireless sensing system where multiple devices collaboratively perform an inference task. In \cite{iotj.10308585}, Jiang \textit{et. al.} proposed a collaborative precoding design for adjacent base stations for providing communication and wide range cooperative sensing services for vehicles. From a localization perspective, in \cite{iotj.9186070}, Gu \textit{et. al.} established a general framework of self-localization and cooperative target detection and developed a hybrid coordinate descent localization algorithm for joint position estimation and cooperative target detection. 


However, most existing cooperative mechanisms predominately focus on maximizing the aggregate system throughput or the total sensing area. Such sum-rate maximization strategies tend to over-exploit devices with superior channel conditions while starving those at the edge, leading to a severe ``wooden barrel effect" and uneven energy depletion across the network. Therefore, a critical gap remains in investigating cooperative protocols that go beyond simple performance maximization to address the systemic resource imbalance among heterogeneous nodes, ensuring sustainable and balanced network operations. This necessitates a mechanism to balance system-wise performance with individual device fairness, motivating the fairness-aware cooperative sensing strategy proposed in Chapter 4.


\subsection{Integrated Sensing, Communications, and Computing}
\label{chap1_ssec_4}

As networks evolve towards intelligent ecosystems, the integration of ISAC with other emerging technologies has attracted significant attention. In \cite{tvt.9733335}, Liu \textit{et al.} proposed an intelligent reflecting surface (IRS)-aided ISAC system operating in the terahertz band to maximize the system capacity. For multi-UAV scenarios, in \cite{tvt.9963915}, Zhang \textit{et al.} studied an ISAC-enabled multiple unmanned aerial vehicle cooperative detection scenario. Leveraging ISAC for learning, in \cite{tvt.9792281}, Liu \textit{et al.} studied a vertical federated edge learning system for collaborative objects motion recognition by exploiting the distributed ISAC. In \cite{tvt.huang2022integrated}, Huang \textit{et. al.} investigated ISAC assisted mobile edge computing and leveraged IRS to improve the performances of radar sensing and mobile edge computing. In the digital twin context, ISAC has been combined with the digital twin to address the problem of task scheduling and resource allocation in vehicular edge computing in \cite{tvt.9982429}. 

Parallel to ISAC evolution, MEC has become a cornerstone for low-latency services. MEC has attracted a wide research interests owing to its great potential in reducing latency and energy consumption for executing computation-intensive tasks on resource-limited devices \cite{twc2.10207705,twc2.9177293,twc2.8771176}. To further enhance MEC performance, exploiting NOMA to facilitate task offloading in MEC has been investigated in numerous studies. In \cite{twc2.sheng2020delay}, Sheng \textit{et. al.} investigated the computation offloading in multi-carrier NOMA-enabled MEC under the differentiated uploading delay. Dealing with imperfect CSI, the resource allocation for the NOMA-MEC network with imperfect channel state information has been investigated in \cite{twc2.9353556}. In \cite{twc2.ding2022hybrid}, a general hybrid NOMA-MEC offloading strategy has been proposed. 

Beyond relying solely on edge servers, the multi-tier MEC paradigm that allows the cooperation among edge and cloud servers enables a more efficient and flexible utilization of the computation resources across different tiers of the networks. In \cite{twc2.wang2023adaptive}, Wang \textit{et. al.} studied a joint edge video transcoding and client video enhancement optimization problem for adaptive bitrate streaming in multi-tier wireless computing networks. In \cite{twc2.xu2023federated}, Xu \textit{et. al.} investigated the optimization mechanism for accelerating the federated learning enabled two-tier computing based on the fully-decoupled radio access network architecture. 

Recently, several efforts have been devoted to leveraging ISAC for MEC to achieve the deep integration of sensing and task offloading. In \cite{tvt.9828481}, Qi \textit{et. al.} presented a framework for the general integration of three isolated functions of sensing, computing, and communication to explore their relationship. To exploit coordination gains, in \cite{twc2.zhao2022radio}, Zhao \textit{et. al.} proposed a wireless scheduling architecture to exploit the coordination gains of sensing, communication, and computing. Focusing on artificial intelligence inference, In \cite{twc2.wen2023task}, Wen \textit{et. al.} proposed a task-oriented ISCC scheme for edge artificial intelligence inference. In \cite{twc2.10415206}, an edge intelligence-oriented ISAC has been proposed, in which multiple base stations can offload the sensing data to the edge server for model training. 


Despite the initial explorations in ISCC, most existing works treat sensing and computing as loosely coupled add-ons, failing to address the complex interference dynamics in multi-tier architectures. Specifically, the literature lacks a closed-loop resource management framework that can effectively mitigate the inter-functionality interference while orchestrating the synergy between sensing data acquisition and subsequent computation tasks. It remains an open issue regarding how to minimize system-wide energy consumption through cross-tier coordination, which motivates the two-tier integrated sensing and task offloading framework proposed in Chapter 5.

\section{Thesis Contributions}
\label{chap1_sec_contribution}

Motivated by the research gaps identified in the literature review, this thesis investigates four resource-efficient designs to systematically address the resource bottlenecks in ISAC systems across the time, frequency, spatial, and functional domains. We focus on the joint optimization of beamforming, scheduling, and multi-dimensional resource allocation to enhance the overall system-wise resource efficiency. The main contributions are summarized as follows:
\begin{itemize}	
    \item \textit{Breaking Temporal Rigidity via Channel Sharing-Aided ISAC (Chapter 2):} To address the inefficiency caused by static time-slot allocation, we investigate an energy-efficient channel-sharing aided ISAC scheme. We propose a dynamic scheduling mechanism where the base station reuses the downlink channels of cellular users to perform multi-target sensing tasks. To balance the trade-off between sensing energy efficiency and communication quality, we formulate a joint optimization problem of sensing scheduling and transceiver beamforming. We solve this non-convex problem by addressing the fractional objective function via Dinkelbach's method, obtaining the beamforming solution through semidefinite relaxation (SDR) and Lagrange duality, and solving the matching-based scheduling problem using a swap-matching algorithm.
    \item \textit{Unlocking Spectral Capacity via NOMA-Aided ISAC (Chapter 3):} To overcome the spectral capacity limit of orthogonal access, we investigate a NOMA-aided ISAC framework. We design a scheme where the base station uses superimposed signals to serve multiple users while simultaneously performing sensing tasks. We formulate an optimization problem to maximize the sensing efficiency, which is a novel metric defined as the number of successfully sensed targets per time unit, by jointly optimizing the beamforming, the NOMA transmission duration, and the sensing scheduling. To tackle the mixed-integer non-convexity, we propose a decomposition-based algorithm. Specifically, the beamforming is optimized via successive convex approximation (SCA), the transmission duration is determined using a bisection-search method based on its monotonic feature, and the sensing scheduling solution is derived via a cross-entropy learning algorithm.
    \item \textit{Enhancing Spatial Fairness via ISAC-Enabled Cooperative Sensing (Chapter 4):} To overcome single-node coverage blind spots and address spatial unfairness, we investigate a fairness-aware ISAC-enabled multi-device cooperative sensing strategy. In this framework, distributed devices access the channel in a time-division manner, with each device performing environmental sensing and data transmission simultaneously during its assigned time slot. We formulate an optimization problem to maximize the fairness-aware system-wise throughput while strictly guaranteeing cooperative sensing quality. We design a block coordinate descent (BCD) algorithm to decompose the problem into beamforming and time allocation subproblems. Specifically, we analyze the necessity of dedicated sensing signals and employ SCA to optimize the beamforming vectors. Furthermore, we characterize the features of the optimal time allocation to derive a semi-analytical expression, ensuring an efficient and balanced resource allocation.
    \item \textit{Bridging Functional Silos via ISCC (Chapter 5):} Finally, to address the conflict of multi-service coexistence and bridge the functional silo between sensing and computing, we investigate a NOMA-assisted integrated sensing and two-tier task offloading system. We propose a framework where an access point supports edge task processing while performing continuous sensing, with additional offloading capabilities to cloudlet servers. To counteract the resource competition between high-power sensing and latency-sensitive computing, we propose a cross-functional synergy strategy that employs tier-specific signaling to decouple mutual interference. A joint optimization problem of transmit beamforming, two-tier dedicated sensing signals, computation offloading strategies and associated allocations of the communication and computing resources is formulated, with the objective of minimizing the total energy consumption. We exploit the structural features of the problem and propose a decomposition-based algorithm to solve it, realizing a holistic functional closed-loop design that supports computation-intensive applications alongside ubiquitous sensing.
\end{itemize}


\section{Statement of Originality}
\label{chap1_sec_Originality}
All the contributions listed in Section \ref{chap1_sec_contribution} are published with our own originality. The main results presented in this thesis are included in the following papers.

\begin{enumerate}[ ~~~~{[}1{]} ]
    \item \textbf{Chenglong Dou}, N. Huang, Y. Wu, L. Qian, and T.Q.S. Quek, ``Channel Sharing aided Integrated Sensing and Communication: An Energy-Efficient Sensing Scheduling Approach,” \textit{\textbf{IEEE Transactions on Wireless Communications}}, vol. 23, no. 5, pp. 4802-4814, May 2024. 
    \item \textbf{Chenglong Dou}, N. Huang, Y. Wu, L. Qian, and T.Q.S. Quek, ``Sensing-Efficient NOMA-aided Integrated Sensing and Communication: A Joint Sensing Scheduling and Beamforming Optimization,” \textit{\textbf{IEEE Transactions on Vehicular Technology}}, vol. 72, no. 10, pp. 13591-13603, Oct. 2023. 
    \item \textbf{Chenglong Dou}, N. Huang, Y. Wu, L. Qian, Z. Shi, and T.Q.S. Quek, ``Integrated Sensing and Communication Enabled Multi-Device Multi-Target Cooperative Sensing: A Fairness-aware Design,” \textit{\textbf{IEEE Internet of Things Journal}}, vol. 11, no. 17, pp. 29190-29201, Sept. 2024. 
    \item  \textbf{Chenglong Dou}, M. Dai, N. Huang, Y. Wu, L. Qian, and T.Q.S. Quek, ``Integrated Sensing and Two-Tier Task Offloading via Non-orthogonal Multiple Access: An Energy-Minimization Design,” \textit{\textbf{IEEE Transactions on Wireless Communications}}, vol. 23, no. 12, pp. 19157-19171, Dec. 2024. 
\end{enumerate}